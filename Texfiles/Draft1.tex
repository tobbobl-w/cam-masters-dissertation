\documentclass[12pt]{article}
\usepackage{graphicx}
\usepackage{amsmath}
\usepackage{amssymb}
\usepackage[authoryear]{natbib}
\usepackage{enumerate}
\setlength{\parindent}{0pt}
\usepackage[parfill]{parskip}
\date{Lent 2023}
\title{Dissertation draft}
\author{Tobias Leigh-Wood}
\usepackage[letterpaper, margin=1.1in]{geometry}
\linespread{1.1}

\begin{document}
\maketitle


Standard economic theory suggests that retirement annuities should be highly prized by individuals as a way to
insure against the risk of late death \cite{yaari_65}. However, in developed countries rates of annuitization are far below the
levels that theory predicts. In this paper I test two competing hypotheses for the annuity problem: bequests and
pessimistic life expectancy. In most developed countries, individuals are forced to contribute to either a defined
benefit or defined contribution pension. A defined benefit pension accumulates throughout your job tenure as
Under the coalition government in the UK the law regarding the use of private defined
contribution pensions changed. Individuals were no longer forced to annuitise their pension pots and could access
them in a variety of ways such as lump sum withdrawals or income drawdown and subsequently the number of annuities
sold in the UK dropped precipitously.


Depending on the reason for the lack of annuitization in the UK, the consumption response of retirees to the pension
reform will differ. If individuals do not annuitise because of pessimistic life expectancy I will show that their
consumption should increase. If, on the other hand, individuals do not annuitise because of a bequest motive, consumption
should not change much as a result of the reform. I will solve lifecycle models for both of these cases and simulate
consumption decisions with, and without, forced annuitization. I will then use a variety of empirical models to measure
the consumption change in early retirement that resulted from the policy reform. The size and magnitude of this change
will be indicative of the mechanism causing the annuitization problem.

The importance of retirement policy to individuals in the UK is growing. The number of individuals of
pensionable age is expected to grow from 11.9 million in 2020 to 15.2 million in
2045 according to the latest ONS statistics and for every 1000 people of working age there will be 341 of pensionable
age in 20145 compared to 280 in 2020 \cite{ons_population_predictions_2020}. The increase in absolute and relative
numbers of elderly makes retirement policy more important. Moreover, private, defined contribution (DC), pensions are
becoming increasingly common and are predicted to grow as current cohorts age \cite{cribb_karjalainen_ifs_2023}.
Therefore, policies regarding how private pensions can be accessed will have a larger impact on overall welfare for
retirees.

The so called "pensions freedom act" received Royal Asset in December 2014 marked the end of a series
of pension reforms carried out by the coalition government between 2010 and 2015. The reform was announced
in the Spring budget and made it possible to withdraw money from a private pension pot subject to the
marginal rate of income tax that an individual faced. In the June 2010 budget the government made a first
reform to the annuitization rules, creating an minimum income requirement above which individuals would not need to
annuitise more \cite{finance_act_hmt_2011}. However, this was set at £20,000 and therefore few individuals
were eligible. The minimum income requirement was scrapped in the 2014 bill finally eliminating the
compulsory annuities market. The impact of the reforms on annuity demand has been documented by \cite{cannon_et_al_nier_2016}.
Using data from the Association of British Insurers they show that annuity demand dropped by 75\% from
its maximum.

\subsection{Literature review}
My paper draws on three main strands of literature. The annuity problem, the retirement saving
problem and lifecycle models. \textbf{this intro is rubbish}

\cite{yaari_65} was the first to show that under standard assumptions we would expect individuals to
annuitise all of their wealth at retirement to insure against the risk of long life. Since then there
has been much literature discussing possible reasons that people do not annuitise. \cite{finkelstein_porteba_2002}
and \cite{finkelstein_porteba_2004} find evidence of adverse selection, thereby making the 'money's worth'
of annuities lower for the general population as opposed to the population of annuitants. However, they
also find that theory would still predict annuitization.

\cite{friedman_warshawsky_qje_1990} show that annuitization decisions can be fully explained by a mixture of
bequest motives and actuarily unfair annuities. They solve an augmented life-cycle model with a range of
parameters on how severe the rate of return is on the annuity versus market rates. For plausible values
they find that individuals would optimally not annuitise much wealth. Similarly to \cite{finkelstein_porteba_2004},
\cite{friedman_warshawsky_chicago_1988} show that there is a significant difference between the life expectancy of
annuitants and the general population in the American annuity market but this cannot fully explain the annuitization
problem. Only when bequest motives are added to the model can annuitization rates be rationalized.

\cite{lockwood_red_2012} builds on this and shows that a realistic bequest motive in lifecycle simulations achieves
realistic annuitization rates. He solves a simple lifecycle model with bequest motives taken from several recent papers
in the literature. The bequest motives he picks therefore match other important aspects of the lifecycle model such
as how much individuals actually bequest and how rich individuals are when they bequest.

\cite{lockwood_aer_2018}

\cite{vidalmelia_lejarragagarcia_munich_2004} have some interesting results. Need to talk about that.


\subsection{Data}
The main data set I use is the English Longitudinal Study of Ageing (ELSA) \cite{main_elsa_citation}. ELSA repeatedly interviews
individuals over the age of 50 and asks them a range of questions relating to their income and wealth as well as expectations about
the future. Importantly it also includes detailed information pensions including the type of pension that an individual holds when
working, thus I am able to distinguish between individuals who have defined benefit and defined contribution pensions. For ease of access
I use Harmonized ELSA \footnote{"This analysis uses data or information from the Harmonized ELSA dataset and Codebook, Version G.2 as of
      July 2021 developed by the Gateway to Global Aging Data. The development of the Harmonized ELSA was funded by the National
      Institute on Aging (R01 AG030153, RC2 AG036619, R03 AG043052). For more information,
      please refer to https://g2aging.org/.”} which ensures that variables are comparable across waves. Since this only includes a
subset of the questions in ELSA I also supplement it with variables taken directly from the data.

I also use ONS life tables data.

I also use annuitant life tables data.

I also use FCA product data.



\subsection{Methodology}

I first solve a modified retirement lifecycle model. The problem that retirees face is as follows:
\textbf{This is not correct -- need to include probs}
\begin{equation*}
      E(U) = \sum_{t = x}^{110 - w}[u(c_{t})]
\end{equation*}
subject to their budget constraint
\begin{equation*}
      c_{t} =a_{t}(1 +r) -  a_{t+1} + y_{t}
\end{equation*}
where $a_{t}$ are asset holdings in time t and $y_{t}$ is income in period 2. Income can come either
from state pensions, defined benefit pension plans or purchased annuities.
\begin{equation*}
      u(c_{t}) = \frac{c_{t}^{1 - \sigma}}{1 - \sigma}
\end{equation*} In some specifications retirees can leave bequests, I use the bequest function from
\cite{lockwood_red_2012}.
\begin{equation*}
      u(b_{t}) = (\frac{m}{1 - m})^{\sigma}  \frac{((\frac{m c_{0}}{1 - m}) + b_{t})^(1 - \sigma)}{(1 - \sigma)}
\end{equation*}
Where $b_{t}$ is the amount left at death, $m$ is a measure of bequest motive strength and $c_{0}$ is
a minimum amount of consumption that individuals want. \textbf{Check this}

I solve the retirees problem using backward induction. At age 110 there is certainty of death so any leftover assets
are carried over into the next period and bequested. This means that the value at the end of the final period is
either 0 (if we do not allow a bequest motive) or the value of bequests. I then take this value function and solve
an individuals final period problem, choosing assets next period (i.e. those to bequest) and how much to consume.

Using the optimal policy function in the last period, I calculate the value of the last period, which the utility
function and the value function evaluated at the maximum. This is then used in the problem the year before that.
I repeat this process back to the age of retirement to obtain optimal consumption amounts for each year of retirement
and associated value functions.

To simulate the ELSA data I solve this retirement problems for each new retiree in the data set dependent on their
objective probability of death each period. I estimate with subjective life expectancies and objective life expectancies,
I also estimate the model both with and without a bequest motive which was picked to fit the unforced real annuity
rates seen in the data. I then estimate several empirical models with the simulated data.

\subsection{Empirical models}

In this section I outline the key empirical models I run with both the simulated consumption data from the
lifecycle models and with the real data from ELSA. I then see which lifecycle model better fits the consumption
response that happened as a result of the pension reform.

I use a regression discontinuity design where I compare the consumption of recent retirees after the policy reform
to recent retirees before the policy reform. The individuals who retired in 2014 and later were not forced to
annuitise their defined contribution pension. The key assumption implicit in regression discontinuities is
that nothing else changes at the time of the jump apart from the policy of interest. And that the policy occurs
without individuals predicting it. The policy change was widely seen as a surprise by the media and financial planners,
some of whom in fact complained that they had not been consulted enough. \textbf{How can I justify no other jumps at the jump}

Retirement year is the running variable and individuals are treated if retirement year is greater than 2013 and less than 2016.
I consumption of individuals up to 2 years into retirement so that the sample size is larger. So if someone retired in 2014 and
had consumption data in 2014 and 2016 I include both values. An individual is considered not treated if they retire before 2013 and after 2011.


The estimating equation is therefore

\subsection{Rough plan}
\begin{itemize}
      \item Intro
            \begin{enumerate}
                  \item I think Eric French wrote something about population that I could use in then intro to say why it is important.
                  \item "Latest data from HM Revenue Customs (published in April) showed more than £45bn has been taken from pots since 2015."
                        https://www.ftadviser.com/pensions/2022/02/08/pension-freedoms-were-they-really-a-good-idea/


                  \item Add bit about DC/DB pensions in the intro. Also talk about heterogeneity across countries.
                        Some countries want to move towards more annuitisation. This annual review is a good source of info
                        \cite{banks_crawford_ar_2022}
            \end{enumerate}
      \item Lit review
      \item Models
      \item Empirical
            \begin{enumerate}
                  \item Diff in diff
                  \item RDD
                  \item Matching?
                  \item can I use anything I learnt in panel?
                        In some sense the decision to annuitise is a discrete choice problem so I could use something from there.
            \end{enumerate}
      \item Conclusion
\end{itemize}


\bibliographystyle{chicago}
\bibliography{references}
\end{document}