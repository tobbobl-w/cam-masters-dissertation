\documentclass[12pt]{article}
\usepackage{graphicx}
\usepackage{amsmath}
\usepackage{amssymb}
\usepackage[authoryear]{natbib}
\usepackage{enumerate}
\usepackage{booktabs}
\usepackage{pdflscape}
\setlength{\parindent}{0pt}
\usepackage[parfill]{parskip}
\date{Lent 2023}
\title{Dissertation draft}
\author{Tobias Leigh-Wood}
\usepackage[letterpaper, margin=1.1in]{geometry}
\linespread{1.1}

\begin{document}
\maketitle


Standard economic theory suggests that retirement annuities should be highly prized by individuals as a way to
insure against the risk of late death \cite{yaari_65}. However, in developed countries rates of annuitization are far below the
levels that theory predicts.
Under the coalition government in the UK the law regarding the use of private defined
contribution pensions changed. Individuals were no longer forced to annuitise their pension pots and could access
them in a variety of ways such as a lump sum withdrawal on retirement or income drawdown and subsequently the number of annuities
sold in the UK dropped precipitously.

I first use the policy reform as a discontintuity to measure the impact of forced annutisation on the consumption
of individuals in the first few years of retirement. Given that the reform was implemented suddenly and without
advanced notice before the Spring 2014 budget I claim that individuals with a retirement year in 2015, 2016 or 2017
are otherwise similar to those with a retirement year in 2011, 2012 and 2013 but the do not need to annuitise their
defined contribution pension pots.


In this paper I test two competing hypotheses for the annuity problem: bequests and
pessimistic life expectancy. Depending on the reason for the lack of annuitization in the UK, the consumption response of retirees to the pension
reform will differ. If individuals do not annuitise because of pessimistic life expectancy I will show that their
consumption should increase. If, on the other hand, individuals do not annuitise because of a bequest motive, consumption
should not change much as a result of the reform. I will solve lifecycle models for both of these cases and simulate
consumption decisions with, and without, forced annuitization. I will then use a variety of empirical models to measure
the consumption change in early retirement that resulted from the policy reform. The size and magnitude of this change
will be indicative of the mechanism causing the annuitization problem.


The importance of retirement policy to individuals in the UK is growing. The number of individuals of
pensionable age is expected to grow from 11.9 million in 2020 to 15.2 million in
2045 according to the latest ONS statistics and for every 1000 people of working age there will be 341 of pensionable
age in 20145 compared to 280 in 2020 \cite{ons_population_predictions_2020}. The increase in absolute and relative
numbers of elderly makes retirement policy more important. Moreover, private, defined contribution (DC), pensions are
becoming increasingly common and are predicted to grow as current cohorts age \cite{cribb_karjalainen_ifs_2023}.
Therefore, policies regarding how private pensions can be accessed will have a larger impact on overall welfare for
retirees.



\subsection{Literature review}
My paper draws on three main strands of literature. The annuity problem, the retirement saving
problem and lifecycle models. \textbf{this intro is rubbish}

\cite{yaari_65} was the first to show that under standard assumptions we would expect individuals to
annuitise all of their wealth at retirement to insure against the risk of long life. Since then there
has been much literature discussing possible reasons that people do not annuitise. \cite{finkelstein_porteba_2002}
and \cite{finkelstein_porteba_2004} find evidence of adverse selection, thereby making the 'money's worth'
of annuities lower for the general population as opposed to the population of annuitants. However, they
also find that theory would still predict annuitization.

\cite{friedman_warshawsky_qje_1990} show that annuitization decisions can be fully explained by a mixture of
bequest motives and actuarily unfair annuities. They solve an augmented life-cycle model with a range of
parameters on how severe the rate of return is on the annuity versus market rates. For plausible values
they find that individuals would optimally not annuitise much wealth. Similarly to \cite{finkelstein_porteba_2004},
\cite{friedman_warshawsky_chicago_1988} show that there is a significant difference between the life expectancy of
annuitants and the general population in the American annuity market but this cannot fully explain the annuitization
problem. Only when bequest motives are added to the model can annuitization rates be rationalized.

\cite{lockwood_red_2012} builds on this and shows that a realistic bequest motive in lifecycle simulations achieves
realistic annuitization rates. He solves a simple lifecycle model with bequest motives taken from several recent papers
in the literature. The bequest motives he picks therefore match other important aspects of the lifecycle model such
as how much individuals actually bequest and how rich individuals are when they bequest.

\cite{lockwood_aer_2018}

\cite{vidalmelia_lejarragagarcia_munich_2004} have some interesting results. Need to talk about that.

\textit{There are some more papers to include here. Time to go and have dinner. }

\section{Data and Policy reform}

\subsection{Data}

The main data set I use is the English Longitudinal Study of Ageing (ELSA) \cite{main_elsa_citation}. ELSA repeatedly interviews
individuals over the age of 50 and asks them a range of questions relating to their income and wealth as well as expectations about
the future. One benefit of using ELSA is that it includes data on pension type for individuals who are working. Therefore,
I can differentiate between individuals who have a defined contribution pension and those who have a defined benefit contribution.
Importantly it also includes information on pension size that the Institute for Fiscal Studies calculated, this is only available
up until wave 5 at which point I use a real return of 3\% to predict forward the pension wealth variable until retirement.
ELSA also includes a measure of all non-housing financial wealth which I use in some specifications because of these issues with
pension wealth. Because of slight differences between years in some of the ELSA questions I use 'Harmonized ELSA'
\footnote{"This analysis uses data or information from the Harmonized ELSA dataset and Codebook, Version G.2 as of
  July 2021 developed by the Gateway to Global Aging Data. The development of the Harmonized ELSA was funded by the National
  Institute on Aging (R01 AG030153, RC2 AG036619, R03 AG043052). For more information,
  please refer to https://g2aging.org/.”} which ensures that variables are comparable across waves. Since this only includes a
subset of the questions in ELSA I also supplement it with variables taken directly from the data such as questions that deal
with life expectancy.

ELSA also includes questions on expenditures. In particular individuals are asked how much they consume on a range of broad
categories including in-house food consumption, out-of-house food consumption, leisure consumption, clothes consumption and
consumption on utility bills and rent. I use also use two questions on life expectancy that I discuss in more detail below.

I also use life tables from the UK's Office for National Statistics. These give us risk of death for each age group. I adjust them
to make death certain at age 110 as is common in the literature. I transform these so that I have risk of death conditional on being
a given age since this is what is used in the life cycle simulations.


To illustrate the effect the reform had on sales of annuities in the UK I obtained product data from the Financial Conduct Authority.
These track the sale of different financial products overtime including data on annuity sales.


% Plot showing the drop of in annuity sales 

To calculate subjective life expectancies I follow \cite{odea_sturrock_rest_2023}. Individuals are asked “What are the chances that you will
live to be age X or more?” where X changes depending on the age of the interviewee. If individuals were under 65 then X was 75, if individuals
were 66 and older they were asked the age that was 11 to 15 years older than them and is a multiple of 5. From wave three respondents were
also asked “What are the chances that you will live to be age 85 or more?” if they were under 70. As most recent retirees are under 70 we
therefore have two data points. I carry out the following procedures: drop any individuals who think it is more likely they reach an older
age than a younger age since; add, as a third data point their objective chance of reaching 110 according to the ONS life tables; fit these
three points to a Weibull distribution using non-linear least squares. Then I create subjective survival tables.

\textbf{I could go into more detail here? Maybe I should. Add equation etc}


\subsection{Policy reform}

Successive governments in the UK have attempted to reform both the public and private pension system. Prior
to 1987 participation in private schemes was limited to employees of firms that had offered one, and there
were few alternatives to the public state pension or defined benefit scheme that a public sector employee
would offer. From 1989, individuals in the UK were allowed to open tax advantaged self-invested personal pensions
alongside any pension their employer offered. The largest reform prior to 2014 came in the 2004 Finance Act which
rationalised taxation rates on different types of pensions. However, defined contribution pension pots could still
only be used to buy an annuity after taking out a maximum of 25\% as a tax free lump sum withdrawal.

Announced in the Spring budge of 2014, the so called "pensions freedom act" received Royal Asset in December 2014 marked the end of a series
of pension reforms carried out by the coalition government between 2010 and 2015. The reform was announced
in the Spring budget and made it possible to withdraw money from a private pension pot subject to the
marginal rate of income tax that an individual faced. In the June 2010 budget the government made a first
reform to the annuitization rules, creating an minimum income requirement above which individuals would not need to
annuitise more \cite{finance_act_hmt_2011}. However, this was set at £20,000 and therefore few individuals
were eligible. The minimum income requirement was scrapped in the 2014 bill finally eliminating the
compulsory annuities market. The impact of the reforms on annuity demand has been documented by \cite{cannon_et_al_nier_2016}.
Using data from the Association of British Insurers they show that annuity demand dropped by 75\% from
its maximum.

After the announcement in spring 2014 sales of annuities dropped massively and income drawdown products increased.
However, pension pots could also be taken out in one go and this became a much more common option for many people.

Figure \ref{fig:annovertime} shows purchases of annuities overtime and demonstrates the sharp decrease
in purchases that happened from 2014 to 2015. Likewise there was an increase in the number of pensions that were
being accessed using an income drawdown product, but these do not fully account for the drop in annuitisation. So,
either they are withdrawn at once or they are left alone.



\begin{figure}[h]
  \caption{Pension pots accessed}
  \centering
  \includegraphics[width=0.7\columnwidth]{figures/annuity_overtime.pdf}
  \label{fig:annovertime}
\end{figure}


Figure \ref{fig:ann2122} shows the current distribution of how pension pots are first accessed at retirement.
In 2021, 196,736 pots were fully withdrawn at retirement, accounting for over 50\% of pots, prior to the policy
reform this was not the case and most defined contribution pensions were accessed through annuities.



\begin{figure}[h]
  \caption{How pension pots are accessed}
  \centering
  \includegraphics[width=0.7\columnwidth]{figures/annuity_pot_sizes.pdf}
  \label{fig:ann2122}
\end{figure}

Also of specific interest to the annuity market was the European Union's
'Equal Treatment in Goods and Services Directive' of 2004. This prohibited discrimination
in the provision and cost of good and services based on sex. Up until 2011 there was a clause that
stated insurers were allowed to charge different premiums if it was based on evidence that sex is
correlated with different amounts of risk. However, in March 2011 the European Court of Justice
ruled that insurers were not allowed to charge different amounts and gave them until December 2012
to implement the ruling. This change meant that annuity products could no longer be priced differently
for men and women in the UK. However, given this change was implemented two years before the pension
freedoms act, I can still identify the impact of the pensions freedom act on early retirement consumption.

\subsection{Covariate Balance}

Table \ref{tab:sum_stats} shows various summary statistics for each variable of interest from ELSA and the ONS.
First note that the second retirement group is small with just 728 non missing observations for gender as opposed
to 941 individuals in the control group. The control group are less female, retired at a slightly younger age and expected to
retire slightly earlier. This could be because this period was affected by the increase in the state pension age for
women from age 60 in 2010 to age 65 in 2018. Since this change happened gradually and occured over the whole period
I do not expect it to influence the results. The DifferenceAge row tracks the difference between expected and actual
pension age.

As required, retirement year is between 2011 and 2013 for the treatment group and 2015 and 2017 for the control group.
However, as discussed above their interview date is after or the same as the retirement year because we are tracking
consumption in retirement.

The treatment group has higher financial wealth with a median of £65,000 at time of interview as opposed to £55,000
in the control group. Likewise, the second group are more likely to have held at defined contribution pension at some point
and have much more money in them. Both groups are equally likely to have a defined benefit pension with roughly 44\% of
individuals across both samples having a defined benefit pension. Individuals are much more likely to have a DB pension
than a DC pension which reflects the fact that retirees now are still more likely to have a DB pension although the number
with DC pensions are growing. Home ownership is roughly equal across groups although the second group have slightly more
housing wealth.

Both groups are objectively similarly long-lived according when using ONS life tables to calculate life expectancy
based on gender, age and the year the interview was carried out in. Subjective life expectancies are also similar
across groups, with individuals expecting to live another 21 years as opposed to the 24 that the ONS would expect
them to live. \textbf{Maybe add in prob of survival to age 85 or 90 or something}
\textbf{Also add in an indicator for living alone or living with someone else}

Unfortunately, some consumption data is missing for some individuals. The food categories have the least missing data
and the leisure consumption category has the most missing data. \textbf{Check why total consumption adds up to so much
  more than all the categories added together? Do they include rent and housing in consumption?}


Overall, the groups do appear to be slightly different with the treatment group being richer and slightly more likely
to have DC pension wealth. On key demographic characteristics though the groups are similar, the difference in retirement
age between the two groups can be mostly explained by the difference in expected retirement age meaning that individuals
have not on-mass decided to delay retirement and avoid annuitisation.




\begin{landscape}

  \begin{table}

\caption{Summary statistics \label{tab:sum_stats}}
\centering
\begin{tabular}[t]{lllllllllrrll}
\toprule
\multicolumn{1}{c}{ } & \multicolumn{2}{c}{Max} & \multicolumn{2}{c}{Mean} & \multicolumn{2}{c}{Median} & \multicolumn{2}{c}{Min} & \multicolumn{2}{c}{Non Missing} & \multicolumn{2}{c}{Sd} \\
\cmidrule(l{3pt}r{3pt}){2-3} \cmidrule(l{3pt}r{3pt}){4-5} \cmidrule(l{3pt}r{3pt}){6-7} \cmidrule(l{3pt}r{3pt}){8-9} \cmidrule(l{3pt}r{3pt}){10-11} \cmidrule(l{3pt}r{3pt}){12-13}
 & Control & Treat & Control & Treat & Control & Treat & Control & Treat & Control & Treat & Control & Treat\\
\midrule
Gender & 1.0 & 1.0 & 0.455 & 0.493 & 0.0 & 0.0 & 0.0 & 0.0 & 941 & 728 & 0.498 & 0.500\\
RetirementYear & 2013.0 & 2017.0 & 2011.878 & 2015.865 & 2012.0 & 2016.0 & 2011.0 & 2015.0 & 941 & 728 & 0.763 & 0.744\\
InterviewYear & 2016.0 & 2019.0 & 2013.306 & 2017.357 & 2014.0 & 2017.0 & 2011.0 & 2015.0 & 941 & 728 & 1.188 & 1.140\\
YearsSinceRetirement & 2.0 & 2.0 & 1.040 & 0.999 & 1.0 & 1.0 & 0.0 & 0.0 & 941 & 728 & 0.738 & 0.757\\
RetiredAge & 82.0 & 81.0 & 62.918 & 63.750 & 63.0 & 64.0 & 47.0 & 50.0 & 941 & 728 & 4.441 & 4.454\\
\addlinespace
ExpectedRetiredAge & 120.0 & 120.0 & 62.282 & 62.639 & 60.0 & 60.0 & 54.0 & 50.0 & 724 & 609 & 5.320 & 6.206\\
DifferenceAge & 48.0 & 57.0 & -0.682 & -1.205 & -1.0 & -1.0 & -8.0 & -22.0 & 724 & 609 & 4.423 & 5.854\\
FinancialWealth (thousands) & 1910.5 & 2890.0 & 120.487 & 167.023 & 54.9 & 65.6 & -77.0 & -33.0 & 912 & 721 & 210.588 & 294.627\\
DCPension & 1.0 & 1.0 & 0.197 & 0.257 & 0.0 & 0.0 & 0.0 & 0.0 & 941 & 728 & 0.398 & 0.437\\
DCValue (thousands) & 8171.7 & 18367.9 & 73.920 & 128.958 & 0.0 & 0.0 & 0.0 & 0.0 & 832 & 598 & 462.933 & 1081.311\\
\addlinespace
DBPension & 1.0 & 1.0 & 0.438 & 0.434 & 0.0 & 0.0 & 0.0 & 0.0 & 941 & 728 & 0.496 & 0.496\\
OwnsHouse & 1.0 & 1.0 & 0.872 & 0.890 & 1.0 & 1.0 & 0.0 & 0.0 & 938 & 727 & 0.334 & 0.313\\
HouseValue (thousands) & 1300.0 & 3000.0 & 232.313 & 310.218 & 200.0 & 260.0 & 0.0 & -143.0 & 941 & 728 & 181.210 & 265.923\\
ObjectiveLifeExp & 38.0 & 38.6 & 23.840 & 23.721 & 23.9 & 23.4 & 7.9 & 8.8 & 941 & 728 & 4.472 & 4.483\\
SubjectiveLifeExp & 36.6 & 37.9 & 20.902 & 21.129 & 21.2 & 21.3 & 4.6 & 4.1 & 638 & 463 & 6.411 & 6.254\\
\addlinespace
TotalConsump & 435526.0 & 4602.3 & 2427.470 & 796.235 & 648.8 & 702.8 & 130.2 & 136.9 & 768 & 304 & 27132.273 & 492.137\\
FoodConsumpInHouse & 440.0 & 400.0 & 80.059 & 80.884 & 70.0 & 70.0 & 1.0 & 1.0 & 930 & 714 & 48.230 & 44.811\\
FoodConsumpOutHouse & 750.0 & 1500.0 & 66.550 & 88.406 & 50.0 & 50.0 & 0.0 & 0.0 & 937 & 722 & 78.103 & 113.338\\
ClothingConsump & 1450.0 & 2000.0 & 87.642 & 83.045 & 40.0 & 40.0 & 0.0 & 0.0 & 825 & 332 & 135.286 & 150.664\\
LeisureConsump & 530.0 & 150.0 & 84.940 & 75.000 & 60.0 & 75.0 & 0.0 & 0.0 & 744 & 2 & 86.702 & 106.066\\
\addlinespace
UtilityConsump & 483.1 & 580.9 & 108.034 & 112.850 & 96.0 & 100.0 & 0.0 & 0.0 & 785 & 308 & 59.960 & 59.392\\
\bottomrule
\end{tabular}
\end{table}


\end{landscape}


As a further test for balance I regress demographic and financial characteristics of individuals on year of birth and the treatment dummy.
We can then see if the treatment groups differ on key characteristics such as financial wealth or retirement age.
For a regression discontinuity to be valid we need the treatment and control groups to be similar along
all other characteristics apart from the treatment variable.

In particular I run:
\begin{equation*}
  Y_{i} = \alpha + \beta Post_{i} + \gamma YOB_{i} + \kappa (Post_{i} \times YOB_{i}) + \epsilon_{i}
\end{equation*}


\begin{table}

\caption{Covariate Balance \label{tab:cov_balance}}
\centering
\begin{tabular}[t]{lll}
\toprule
 & Pt. est. & SE\\
\midrule
Gender & 2.243 & 10.468\\
RetirementYear & -12.528 & 16.127\\
InterviewYear & -11.243 & 24.959\\
YearsSinceRetirement & 9.716 & 16.056\\
RetiredAge & -12.528 & 16.127\\
\addlinespace
AgeAtInterview & -2.811 & 22.725\\
ExpectedRetiredAge & -240.427 & 129.345\\
DifferenceAge & -246.763 & 128.979\\
FinancialWealth (thousands) & -4769.638 & 5410.054\\
DCPension & -13.608 & 8.933\\
\addlinespace
DCValue (thousands) & -1978.144 & 20284.768\\
DBPension & 7.974 & 10.386\\
StatePension & 67.008 & 71.043\\
OwnsHouse & -0.697 & 6.968\\
HouseValue (thousands) & 6630.186 & 4775.236\\
\addlinespace
ObjectiveLifeExp & 13.086 & 30.559\\
SubjectiveLifeExp & 0.351 & 182.705\\
\bottomrule
\end{tabular}
\end{table}

Table \ref{tab:cov_balance} shows these results. We can see that being in the treatment group, con


A key threat to validity is manipulation of the running variable. This is probably quite a threat in our case
as individuals could delay retirement and/or the purchase of an annuity until after the reform. To check whether
this is occuring we see whether the difference between real retirement age and the first expected retirement age
recorded in the data are different for those retiring before 2014 compared to those retiring after 2014. As shown
above in Table \ref{tab:sum_stats} this not the case. However, since we do not have data on the exact purchase
date of the annuity we cannot track whether annuity purchases were delayed because of this. However, we do still
relatively stable annuity demand before the policy reform. If individuals delayed annuity purchases in expectation
of the reform we would see steady declines in purchases before the reform was announced.


\section{Empirical models}

In this section I outline the key empirical models I run with both the simulated consumption data from the
lifecycle models and with the real data from ELSA. I then see which lifecycle model better fits the consumption
response that happened as a result of the pension reform.

I use a regression discontinuity design where I compare the consumption of recent retirees after the policy reform
to recent retirees before the policy reform. The key assumption implicit in regression discontinuities is
that nothing else changes at the time of the jump apart from the policy of interest. And that the policy occurs
without individuals predicting it and changing their behaviour. As I have argued above the demographic information in the
data suggest that individuals did not delay retirement and that there was no delay in annuitisation as shown by the quick
and sudden decline in annuity purchases. Moreover, anecdotal evidence from media and business sources at the time of the
announcement were all surprised by the policy changes, with some financial planners even complaining that they had not been
briefed.


Retirement year is the running variable and individuals are treated if retirement year is greater than 2014 and less than
or equal to 2017. I use consumption data of individuals up to 2 years into retirement so that the sample size is larger.
So if someone retired in 2015 and had consumption data in 2015 and 2017 I include both values. An individual is considered
not treated if they retire before 2013 and after 2011.  To account for the fact
that the reform only impacted individuals who had accumulated defined contribution pension pots I interact the treatment dummy with
an indicator variable signalling whether the individual had ever held a DC pension pot.

Because of the differences in financial characteristics between the groups I add controls for financial wealth, housing wealth
and whether an individual owns their own home. I also run regression discontintuity with s

\begin{equation*}
  Cons_{it} =  \gamma X_{it} + \beta PostReformDC_{it} + \epsilon
\end{equation*}

And $X_{it}$ is a set of controls.





\section{Methodology and Econometric Strategy}

I first solve a modified retirement lifecycle model. The problem that retirees face is as follows.
Every period retirees solve:
\begin{equation*}
  V_{t}(a_{t}, y) = \underset{a_{t+1}, c_{t}}{\max} \{ u(c_{t}) + p_{t}B(a_{t+1}) + \beta(1-p_{t})V_{t+1}(a_{t+1}, y) \}
\end{equation*}
subject to their budget constraint
\begin{equation*}
  c_{t} =a_{
  t}(1 +r) -  a_{t+1} + y
\end{equation*}
where $a_{t}$ are asset holdings in time t, $y$ is constant income for all periods and . Income can come either
from state pensions, defined benefit pension plans or purchased annuities.
\begin{equation*}
  u(c_{t}) = \frac{c_{t}^{1 - \sigma}}{1 - \sigma}
\end{equation*} In some specifications retirees can leave bequests, I use the bequest function from
\cite{lockwood_red_2012}.
\begin{equation*}
  B(a_{t}) = \bigl( \frac{m}{1 - m} \bigr)^{\sigma}  \frac{(\frac{m}{1 - m}c_{0} + a_{t})^{(1 - \sigma)}}{1 - \sigma}
\end{equation*}
Where $b_{t}$ is the amount left at death, $m$ is a measure of bequest motive strength and $c_{0}$ is
a minimum amount of consumption that individuals want. \textbf{Check this}

First, I discretise the state space. I create a grid from 500 to 50,000 incrementing by 500 for income and 1000 to
500,000 incrementing by 1000 for financial assets. I solve the retirees problem using backward induction. At age 110 there is certainty of death so any leftover assets
are carried over into the next period and bequested. This means that the value at the end of the final period is
either 0 (if we do not allow a bequest motive) or the value of bequests. I then take this value function and solve
an individuals final period problem, choosing assets next period (i.e. those to bequest) and how much to consume.

Using the optimal policy function in the last period, I calculate the value of the last period, which the utility
function and the value function evaluated at the maximum. This is then used in the problem the year before that.
I repeat this process back to the age of retirement to obtain optimal consumption amounts for each year of retirement
and associated value functions.

To simulate the ELSA data I solve this retirement problems for each new retiree in the data set dependent on their
objective probability of death each period. I estimate with subjective life expectancies and objective life expectancies,
I also estimate the model both with and without a bequest motive which was picked to fit the unforced real annuity
rates seen in the data. I then estimate several empirical models with the simulated data.

In retirees first year of retirement I allow them to choose to annuitise some of their wealth. In practical terms
this is moving down the asset grid but up the income grid and seeing if the value of being in that position is better
than where the individual is currently. To calculate this trade-off I calculate the annual annuity payment that follows
from a given annuity cost. I calculate this using objective life tables from the ONS using the following equation:

\begin{equation*}
  Ann = \delta * C * \biggl[\sum_{t = Retage}^{110}\frac{1 - p_{t|Retage}}{(1 + r)^{t - Retage}}\biggr]^{-1}
\end{equation*}

Where $C$ is the one-off payment, $\delta$ is a factor that controls the 'money's worth' of annuity and $p_{t|Retage}$
is the probability of death at age $t$ conditional on being age $Retage$. So individuals can move C on the asset grid
for gaining Ann on the income grid for the rest of their lives.



I also simulate the expected change in consumption given forced annuitisation and no annuitisation.
The benefit of simulating an individuals decision is that we can directly compare what they would have
consumed with what they actually consumed.

\subsection{Rough plan}
\begin{itemize}
  \item Intro
        \begin{enumerate}
          \item I think Eric French wrote something about population that I could use in then intro to say why it is important.
          \item "Latest data from HM Revenue Customs (published in April) showed more than £45bn has been taken from pots since 2015."
                https://www.ftadviser.com/pensions/2022/02/08/pension-freedoms-were-they-really-a-good-idea/


          \item Add bit about DC/DB pensions in the intro. Also talk about heterogeneity across countries.
                Some countries want to move towards more annuitisation. This annual review is a good source of info
                \cite{banks_crawford_ar_2022}
          \item https://www.moneysavingexpert.com/savings/pension-freedom/
        \end{enumerate}
  \item Lit review
  \item Models
  \item Empirical
        \begin{enumerate}
          \item Diff in diff
          \item RDD
          \item Matching?
          \item can I use anything I learnt in panel?
                In some sense the decision to annuitise is a discrete choice problem so I could use something from there.
        \end{enumerate}
  \item Conclusion
\end{itemize}


\bibliographystyle{chicago}
\bibliography{references}
\end{document}