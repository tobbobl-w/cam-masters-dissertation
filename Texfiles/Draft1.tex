\documentclass[12pt]{article}
\usepackage{graphicx}
\usepackage{amsmath}
\usepackage{amssymb}
\usepackage[authoryear]{natbib}
\usepackage{enumerate}
\setlength{\parindent}{0pt}
\usepackage[parfill]{parskip}
\date{Lent 2023}
\title{Dissertation draft}
\author{Tobias Leigh-Wood}
\usepackage[letterpaper, margin=1.1in]{geometry}
\linespread{1.1}

\begin{document}
\maketitle


Standard economic theory suggests that retirement annuities should be highly prized by individuals as a way to
insure against the risk of late death. However, in developed countries rates of annuitization are far below the
levels that theory predicts. In this paper I test two competing hypothesises for the annuity problem: bequests and
pessimistic life expectancy. Under the coalition government in the UK the law regarding the use of private defined
contribution pensions changed. Individuals were no longer forced to annuitise their pension pots and could access
them in a variety of ways such as lump sum withdrawrels or income drawdown and subsequently the number of annuities
sold in the UK dropped precipitously.

Depending on the reason for the lack of annuitization in the UK, the consumption response of retirees to the pension
reform will differ. If individuals do not annuitise because of pessimistic life expectancy I will show that their
consumption should increase. If, on the other hand, individuals do not annuitise because of a bequest motive, consumption
should not change much as a result of the reform. I will solve lifecycle models for both of these cases and simulate
consumption decisions with, and without, forced annuitization. I will then use a variety of empirical models to measure
the consumption change in early retirement that resulted from the policy reform. The size and magnitude of this change
will be indicative of the mechanism causing the annuitization problem.

The number of individuals over the age of 65 is expected to increase dramatically in the 21st century.


\subsection{Rough plan}
\begin{itemize}
      \item Intro
      \item Lit review
      \item Models
      \item Empirical
            \begin{enumerate}
                  \item Diff in diff
                  \item RDD
                  \item Matching?
                  \item can I use anything I learnt in panel?
                        In some sense the decision to annuitise is a discrete choice problem so I could use something from there.
            \end{enumerate}
      \item Conclusion
\end{itemize}


\bibliographystyle{chicago}
\bibliography{references}
\end{document}