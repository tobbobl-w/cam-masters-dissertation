\documentclass[12pt]{article}
\usepackage{graphicx}
\usepackage{amsmath}
\usepackage{amssymb}
\usepackage[authoryear]{natbib}
\usepackage{enumerate}
\usepackage{hyperref}

\hypersetup{
    colorlinks,
    citecolor=black,
    filecolor=black,
    linkcolor=black,
    urlcolor=black
}
\setlength{\parindent}{0pt}
\usepackage[parfill]{parskip}
\date{June 2023}
\title{Dissertation reading notes}
\author{Tobias Leigh-Wood}
\usepackage[letterpaper, margin=1.1in]{geometry}
\linespread{1.1}
\begin{document}
\maketitle
\tableofcontents
\newpage

\section{Comparing Retirement Wealth Trajectories on Both Sides of the Pond}
\cite{Blundell_et_al_2016_fs}

\subsection{Introduction}
\begin{itemize}
  \item Compare wealth decumulation in retirement in the US and the UK using HRS and ELSA data.
        ELSA was modelled on HRS so they are comparable. Both follow elderly people and have similar questions
        and methodologies. Data from 2002 to 2010.

  \item Median wealth at ages 70-74 is similar. But UK individuals tend to have slightly higher income.
        They also have lower medical expenditures because of the NHS. But UK households unexpectedly decumulate assets more slowly.
        This is partly due to the rise in house prices in the UK during that period.


\end{itemize}
\subsection{Empirical facts on asset decumulation}
\begin{itemize}
  \item Wealth in the US has more of a right-skew at the start of retirement. Medians are similar.
  \item Primary housing wealth accounts for 2/3rds of UK retirement wealth but only 1/3 of US retirement wealth.
  \item Life expectancies are pretty similar. US household income is higher than the UKs but it drops with age
        whereas UK income rises with age -- this could be down to different indexation practices between the UK and US. UK pensions
        are often indexed above prices.

  \item Median net wealth grew in the UK but fell in the US. Even once we strip out housing wealth assets are decumulated faster
        in America. Non-housing assets are much larger in the US so this could be because in the UK people want to keep enough assets to
        finance funeral costs. But this is still surprising.

\end{itemize}
\subsection{Potential explanations of cross country differences}
\begin{itemize}

  \item Check whether the life-cycle model explains these differences.
  \item \textbf{Bequest motives}
        \begin{itemize}
          \item Empirical literature in the US points to this being a channel. Less of a literature in the UK but
                surveys have shown that people want to pass on wealth.
        \end{itemize}

  \item \textbf{Precautionary savings}
        \begin{itemize}
          \item \cite{de_nardi_et_al_2010} have shown that pre-cautionary saving for medical expenses explain alot of the slow decumulation
                of assets in America.

          \item There is not comparable data for late life medical expenses in the UK. And although the NHS covers some medical
                expenses they do not cover care home costs which can be a large proportion of medical expenses.
          \item \textit{Note to self: } \cite{french_jones_jae_2004} estimate the stochastic process for medical expenses in retirement.

          \item But because there is not enough data we cannot compare the US and the UK.
        \end{itemize}

  \item \textbf{Housing}
        \begin{itemize}
          \item \cite{Nakajima_Telyukova_ier_2020} make a life-cycle model that takes into account the extra utility from housing
                and the fact that it is difficult to borrow against. Without housing assets they find that assets would up to 55\% lower.
                Combination of utility from housing and housing being difficult to get rid of.

          \item Later work has shown that housing is not a pre-cautionary saving tool.
          \item In the US there is a government administered equity release scheme, the UK has several private operators. In 2011 2\%
                of homeowners had some reverse mortgage.

          \item Nature of housing as an asset impacts decumulation. Empirical work has not been able to distinguish mechanisms from each
                other -- consumption value, transaction costs or risk-return matrix.

        \end{itemize}

\end{itemize}

\subsection{Conclusions}
\begin{itemize}
  \item Housing can explain difference between the UK and US. But so far there is not a satisfactory life-cycle model
        that can account for the difference in asset decumulation.
\end{itemize}

\section{Savings After Retirement: A Survey}
\cite{de_nardi_et_al_are_2016}
\subsection{Introduction}
\begin{itemize}
  \item 1/3rd of US wealth is held by those over 65. Standard life-cycle models with known time of death cannot account for the
        slow decumulation of these assets, particularly for those who are wealthy.

  \item Two groups of mechanisms that could explain this. 1. \textbf{Risks} -- uncertainty over time of death and medical expenses.
        2. \textbf{Bequests}.

  \item Both of these have similar implications for saving in old age.
\end{itemize}
\subsection{Facts about retired households}
\begin{itemize}
  \item Three main sources of retirement wealth -- pension, social security and private savings.
  \item To understand how assets are decumulated across retirement \citeauthor*{de_nardi_et_al_are_2016} use
        data from AHEAD and HRS. They split up the data by postretirement permanent income (PI). This is the average nonasset,
        non means tested social security income over retirement.

  \item Mortality bias -- assets in the unbalanced panel are 50\% of assets in the balanced panel at the start of the
        period.

  \item Show an interesting graph of decomposed medical expenses by age and by source of spending. Social spending
        programs make up the difference in out-of-pocked medical spending as opposed to differences in the use of medical services.

  \item Medical spending has a massive right skew. Top 5\% of the distribution account for over a third of the total spend.

  \item Wealthy and the rich have significantly lower mortality and live longer.
  \item The literature is very mixed on bequests -- some papers find that lots of people leave bequests whereas others find no bequests
        for a majority of the population and then large bequests for the wealthiest.

\end{itemize}

\section{HOME EQUITY IN RETIREMENT}
\cite{Nakajima_Telyukova_ier_2020}
\subsection{Overview}
\begin{itemize}
  \item Look at the role of housing equity in the retirement saving problem.
  \item Find, contrary to \cite{de_nardi_et_al_2010}, that medical expenses do not explain the decumulation problem.
  \item This is the closest in content to the paper that I want to write.
\end{itemize}



\section{THE RETIREMENT-SAVINGS PUZZLE REVIEWED: THE ROLE OF HOUSING AND BEQUESTS}
\cite{Suari_Andreu_et_al_jes_2019}

\subsection{Introduction}
\begin{itemize}
  \item Three categories of explanation for the retirement saving problem (RSP) -- uncertain mortality, bequests and precautionary saving.
  \item Parallel literature looking at housing equity in retirement (HER), find that home-owners are unlikely to draw down
        housing equity because they like living in their home, and partial drawdowns are costly.

  \item \cite{Nakajima_Telyukova_ier_2020} augment the model from \cite{de_nardi_et_al_2010}. The NT model treats housing
        as a different asset. They find that home ownership interacts with the bequest motive.
\end{itemize}
\subsection{The retirement-savings problem}
\begin{itemize}
  \item Three strands to the RSP literature. The first looks at the impact of lifetime uncertainty on retirement savings.
        Initiated by \cite{yaari_65} with recent contributions by \cite{de_nardi_et_al_2010}. The risk of outliving their wealth leads
        individuals to decumulate more slowly than they would if time of death was certain.

  \item The second set of literature studies bequest motives. Altruistic, egoistic, and strategic motives.
        There is also a small literature on inter-vivo.

  \item More recently, the impact that uncertain medical expenses can have on retirement savings has been studied.
        \cite{de_nardi_et_al_2010} show that this can explain a large part of the savings of singles.
\end{itemize}

\subsection{Home equity in retirement}

\begin{itemize}
  \item Housing is special because of its role as a consumption and investment good, and because it is infrequently traded due to
        high transaction costs.

  \item There is at most a modest decline in housing wealth among older people. \cite{venti_wise_nber_2004} have shown that this is
        caused by the subset of households that experience widowhood or a nursing home event. They conclude that
        housing wealth is generally not used for consumption; this explains the low demand for reverse mortgages and implies we should
        not count housing wealth as retirement wealth.

  \item Less moving in the UK than the US. Higher taxes and lower variation in climate and institutional choice.
        Confirmed by \cite{Blundell_et_al_2016_fs}.

  \item Few international studies. A couple of dutch papers have shown also that older people tend not to move or use reverse
        mortgages.

\end{itemize}

\subsection{Models of retirement saving with housing}
\begin{itemize}
  \item First key contribution comes from \cite{yang_red_2009}. She builds a model that explicity models housing and its features.
        Owner occupied housing counts contributes to utility and can be used as collateral.

  \item Households can either be owners or renters. They can become renters exogenously or endogenously. She matches the model
        to parameter estimates from the literature and uses it to explain features of the life-cycle pattern of housing and non-housing
        consumption.

  \item She finds that the transaction cost associated with moving makes older households much less likely to sell their house.
        If transaction costs are 0 then households would sell houses at a faster rate.

  \item But the model leaves out key things features that are in the RSP literature; death uncertainty, medical expense uncertainty and
        bequests which could also explain the pattern.

  \item \cite{Nakajima_Telyukova_ier_2020} adds housing consumption to the model from \cite{de_nardi_et_al_2010}.
        Quite a complicated model but I don't see why I could not copy it. \citeauthor*{Nakajima_Telyukova_ier_2020} have a single health
        shock as opposed to a medical expense shock.

  \item NT find that bequest motives and housing are the big drivers of retirement saving. But \cite{de_nardi_et_al_2010} only use
        singles and they do not match the housing distribution in their method of simulated moments. (Is this process, in general, robust
        to choosing different moments?)

\end{itemize}

\subsubsection{Bequest motives}
\begin{itemize}
  \item Authors consider three different motivations for bequests.
  \item \textbf{Altruistic} NT model bequest motvie as an egoistic motive. Households just want to have positive
        wealth at death no matter who they are passing on the bequests to. An altruistic bequester models the utility of the heirs and
        seeks to maximise this. With concave utility, bequests are decreasing in the life time income of the heirs because their marginal
        utility from money is lower.

  \item \textbf{Strategic}. Bequests should be seen in the context of inter-generational transfers where parents care about
        what their children do for them in old age. Re-write utility with a parameter which increases parents utility but decreases the heirs.
        Some evidence for this channel but I think it is a bit shaky. There is also some evidence that inter-vivo transfers increase if someone
        is a caregiver.


\end{itemize}

\subsection{Government policy timeline}
\begin{itemize}
  \item Freedom and choice in pensions announced in March \cite{pen_freedoms_hmt_2014}. Gains royal
        assent in December becoming the Taxation of Pensions Act. \cite{tax_pensions_hmg_2014}

  \item What other reforms were there? 2010 budget annuitisation
\end{itemize}

\section{The Cost of Annuities: Implications for Saving Behavior and Bequests}
\cite{friedman_warshawsky_qje_1990}

\begin{itemize}
  \item Interesting point re annuities being low because of state pension.
        I could test consumption response to forced annuitisation if that is the
        case as well. I think it would just mean increased spending?

  \item Make a good point that other explanations for the AP are to do with unfairly priced
        annuities. This could be subjectively or not and it would have the same impact (or
        a mixture of the two). So I don't think there is massive need in estimating subjective
        survival curves.

  \item I think there has been some work by cannon and tokes on estimating this in the
        early 2010s so I am not sure whether there is point me trying to do this again using the
        CMI annuitants tables.

  \item Also test model with both bequests and unfair annuities. This seems most obvious tbh.
        Maybe whole idea is rubbish because of this.

  \item Could I compare how a mixture of load factors vs a mixture of subjective death probs
        vs a mixture of adverse selection interacts. Disentangling the extent to which each causes
        people not to annuitise would be difficult though. But can use evidence from ELSA and CMI for
        subjective and adverse selection amounts respectively.

  \item \begin{equation*}
          \mathbb{E}(U) = \sum_{0}^{w - x - 1}[p_{t}U_{t}(c_{t}) + p_{t}q_{t}V_{t+1}(G_{t+1})]
        \end{equation*}
        Where $V_{t+1}(G_{t+1})$ is the utility from bequests next period and $q_{t}$ is the probability of death.


  \item Does some simulations showing how bequests and acturarial unfairness interact with each other.
        This type of table could be a good thing to include in my paper. Then a section showing implied consumption
        changes and then the data work. I think I am kinda close to being able to make that anyway.

  \item What is expected yield differential? D
  \item \textit{" but investigating it seriously would require data on the
          degree of actuarial unfairness in the pricing of annuities not just for
          individuals but for double or even multiple lives."} This is interesting.
        Could I compare annuitisation decisions of couples and singles? Singles
        should be more likely to buy annuities if this is true.

  \item Couples should also have a larger consumption response. Can we solve
        two person models? Probably? But I think joint annuities would face
        more than double the problem with actuarially unfair annuities worse --
        impact would be multiplied and not summed.

  \item Have not seen this before? Could be interesting. I think Eric solved a life-cycle model
        with a couple in so I could try and look at that paper. (No I think it was another paper from S130)

\end{itemize}

\section{Annuity Prices and Saving Behavior in the United States}
\begin{itemize}
  \item Solve life cycle models again. The variable of interest they use is the
        loading factor.

  \item Need bequest motives to get annuity rates that are observed.
  \item "See \cite{friedman_warshawsky_nber_1985} for a brief discussion of several
        of these other possible explanations." that is for other explanations for
        lack of annuitisation. Had a look at this and it wasn't super useful. Just
        seemed like a first draft of the paper.



\end{itemize}


\section{Bequest motives and the annuity puzzle}
\begin{itemize}
  \item Literature (Abel 2004?? Find when landed) claims that bequest motives
        can only explain part of the AP. But Lockwood shows that it can explain why people
        annuitise no wealth given loading factors (prices etc) of annuities.

  \item  (Davidoff et al., 2005??) is another paper on annuities I should read.

        item
\end{itemize}


\section{Personal notes}

\subsection{Saturday before due date}
\begin{itemize}
  \item Thistlethwaite and Campbell, 1960 -- RD literature
  \item Regression Discontinuity Inference with Specification Error -- Lee and Card. Seems like an interesing paper
  \item Should I use financial wealth variables at retirement instead of at interview? I guess in alot of cases I wont
        know this. I could check the differences for the people that I have pre and post for.
  \item I also need to see what happens if I include family size.
  \item Ok data work is done. Now I should need to work on julia data. I think I need to run some more subjectives
\end{itemize}

\subsection{ELSA notes}
\subsection{\textbf{20-07}}
\begin{itemize}
  \item What other ELSA variables to use?
  \item What will home ownership change?
  \item How to deal with top wealth deciles? - at the moment they are excluded because of the grid.
  \item Include other sources of income i.e. Got Dc pension
  \item Run model just with pension wealth and not financial wealth
  \item We could probably generate non annuitisation by adding some fixed cost.
  \item OK, the financial wealth variable I use is quite comprehensive including all non-housing
        wealth which is good.
  \item Deflator?
  \item

  \item Pension wealth is what I need to look at now \textbf{DONE}
  \item What
  \item Do we look at just individuals? Or just households or both?
\end{itemize}

\subsection{\textbf{14-07}}
\begin{itemize}
  \item Pension wealth variables from IFS -- I could illustrate
        the method with just financial wealth and then use pension wealth calculations
        as well. Pension wealth gives us retirement income. This leaves financial wealth to be
        invested in annuities.

  \item Apparently there are pension wealth variables for DC pensions?
        Cant find in harmonized but perhaps they are in the pension files?

  \item Could subjective survival probs also explain slow asset decumulation. And then by the time they are old
        they have medical risks.

  \item "Allowing individuals to solve optimisation problems using hueristics"
\end{itemize}

\subsection{\textbf{12-07}}
\begin{itemize}

  \item Got CMI access.

  \item To do: \begin{enumerate}
          \item Start regression work - done
          \item Finish making subjective survival curves - on their way
          \item Finish lit review - done
          \item Start methodology section - need to do
          \item Write couple code - need to do
          \item Write behaviourel model code need to do
          \item Run lifecycle models with different starting points - this is to do.
                Do we have size of pension though? No. Only size of income.

          \item Do I want to add defined pension income to the model? Yes I think so.
          \item Can also check the consumption responses of individuals who think they are going to leave a bequest.
        \end{enumerate}
\end{itemize}

\subsection{\textbf{11-07}}
\begin{itemize}
  \item Good idea about couples. Good idea about table showing \% of annuititants
        with param values and then also consumption changes.

  \item How to split consumption? Guess have more grids? -- Don't need to since only one
        continuation value and there is no heterogeneity between partners.
  \item How do behaviourel models change the DP problem. Could present bias
        be part of the explanation.

  \item Also introduction of lifetime cap in care costs could be interesting
        hypothetical policy issue to mention.

  \item
\end{itemize}


\subsection{\textbf{08-07}}
\begin{itemize}
  \item Trying to understand the weibull code from \cite{odea_sturrock_rest_2023}. Non-linear
        least squares
  \item Big docs have questions in that we can use.
\end{itemize}

\subsection{\textbf{06-07}}
\begin{itemize}
  \item Do I have death data? check UKDS.
        Harmonized data does have year of death but only for very few
        people. It doesn't have it for the majority of exits.

  \item I can change pension for individuals and asset levels and subjective life expectancy.
  \item

\end{itemize}


\subsection{\textbf{04-07}}
\begin{itemize}
  \item How to make a consumption floor? Is it a question of giving income?
  \item Also check
  \item should think about the impact of housing on the effects of the change
        -- will their be heterogeneity

  \item To do
        \begin{enumerate}
          \item Make julia module
          \item add gender into life cycle
          \item get subjective life expectancies -- have got the code from
                david and cormac but I think they have birth dates and death dates which
                arent included in the normal elsa data.
          \item generate correct rates of annuitisation
        \end{enumerate}
\end{itemize}
\subsection{\textbf{30-06}}
\begin{itemize}


  \item Continue working on retirement lifecycle code.
        Once we have a working version of a simple model we can then attempt
        to add the extensions.

  \item I think I should also check the \cite{odea_sturrock_rest_2023} paper,
        I think they just use a simple model without bequests.

  \item Simple model is working. Now we want to add medical risk?
  \item First clean up code, put it into a function where we can change
        everything easily.

  \item Plot consumption - done. Also sorted out some problems with the code.
  \item Then I need to augment with annuity products -
        I should be able to do this. At age 65 they can make a decision to buy an annuity or not.
        Check expected utility from consumption path.
        Then do the same with bequests.
        Shall I asked ODea and Sturrock for codes?

  \item Lockwood makes a discrete space that is exponential.
        Seems a bit strange tbh. I'd like to look at Odea's code.
        How to draft the email?

  \item So the grid of assets is created from the consumption choices?
        -- Yes I think that is how it works and then consumption should always be in the grid.


  \item Also should compare to income drawdown. Most pensioners have income
        below the taxable amount, and therefore the optimal amount of drawdown
        should not be impacted by taxes.
  \item And state pension?
  \item I guess in it's current

\end{itemize}

\subsection{\textbf{28-06}}
\begin{itemize}
  \item Not sure what I should work ok. I could do more data work or work on the simulation part.
        I think simulation part. Data is probably not far from being done tbh.

  \item Aim is to get a small working version just with risky retirement life expectancy.
  \item
\end{itemize}


\subsection{\textbf{27-06}}
\begin{itemize}
  \item Pretty bad week of work. Understand the data a bit better.
        But have made little progress elsewhere. I want to work on my Julia code
        to replicate the analysis in \cite{odea_sturrock_rest_2023} and \cite{lockwood_red_2012}.

  \item Lets have a look at the matlab code from Lockwood. Is actually kinda hard but I think doable.
        That would be the bit I enjoy the most.
        Why do people keep on coming up with new life-cycle models though.
  \item
  \item What are the implications for the reason no annuities being the returns in the stockmarket.
        - could someone look at actual returns that retirees get. Presumably many switch to portfolios with bonds.

  \item I think the implication would be increased
  \item \cite{inkman_et_al_rfs_2011} -- reading this now. Say that there is an annuity question in the
        Income and Assets section of the ELSA data set. "IAAIM" this variable. Makes sense now how they got
        such a small amount. No way to tell whether people are using income drawdown products or not.
        Guess we could see if non-annuity income increases after.  "IAPPEN" then this variable is the amount of
        pension income which includes annuity income. Not sure whether we should see this go up or down around 2013.

\end{itemize}

\subsection{\textbf{19-06}}
\begin{itemize}
  \item Get familiar with the pension data and the different pension variables. \cite{odea_sturrock_rest_2023} -- in this paper they calculate money's
        worth using subjective mortality risk as opposed to normal life tables. No simulations, just
  \item
\end{itemize}


\subsection{\textbf{18-06}}
\begin{itemize}
  \item Dear Eric, thanks for your advice -- I'll stay clear of trying to estimate a savings model. That paper looks
        very comprehensive so I am not sure what I could add to the topic in four weeks.

        I've come back to the ideas around the reform to annuities and how it fits with the implications that different models
        of retirement saving would predict. Models that have been estimated, such as the Lockwood paper, give quite sharp
        predictions about what should happen to savings and consumption in retirement when the annuity constraint is lifted.

        My analysis could take several of the most recent models in the lifecycle literature that explain retirement saving
        and estimate the expected changes in consumption of retirees in the first few years of retirement conditional on not
        being forced to annuitise pension wealth

  \item If Lockwood paper is correct and bequests are the driver of retirement savings then we should see a decrease in savings
        from income. If individuals dont buy annuities because they aren't priced fairly for them then we should see a rise in consumption
        when annuities are no longer compulsory. One section of the paper could fit these two simple models of non-annuitisation using
        the drop in demand for annuities.


  \item ok found quite an interesting new paper \cite{inkman_et_al_rfs_2011}. Look
        at annuity demand and characteristics that are associated with it. Also,
        estimate a life-cycle model that can explain the portfolio choices
        of individuals in the data.

  \item Mmm could I also compare those with a DB pension and those with a DC pension.
        Nothing changed for DB pension pots. Could I do a diff in diff using the DB pension group
        as the control? I think that would work.

  \item Have a look at the pension data today. Can we distinguish the point at which individuals retire?
        When they convert pension to annuity income? Or when they keep on running it down as normal?

  \item I also need to let Eric know and just run this past him as well.


  \item Does consumption increase with bequest motive as well? Only if people under forced bequests were
        saving more at the start of retirement to regain what they were forced to annuitise.


\end{itemize}

\subsection{\textbf{16-06}}
\begin{itemize}
  \item Have a better understanding of the data.

  \item But I still need a better idea. Read Lee Lockwood RED paper.
  \item Also figure out pre-retirement pension DC pots. i.e. what they can spend
        on annuities.

  \item I think a basic easier question would be what impact has the change had on consumption profiles.
  \item Could I estimate a small model somehow?


  \item New idea is to estimate reduced form estimates of the impact compulsory annuitisation
        had on wealth.

\end{itemize}


\subsection{\textbf{13-06}}

\begin{itemize}
  \item There is data on consumption and spending in ELSA. The technical reports have good information on what is available
        in the data sets.

  \item Questions: Is health expenditure data available? There is some social care cost data but there is not any medical expense data.
        Or I cannot find it at least.

  \item \textit{Questions/to do:}
        \begin{enumerate}
          \item Minimum distance estimation and method of simulated moments in Julia? Solving the model in Julia?
                \begin{enumerate}
                  \item Minimum distance theory vs method of simulated moments -- check Hansen.
                  \item Perhaps get a small working example going and then play around with it. There are some on quant econ that I can use \hyperref{text}
                        https://github.com/fediskhakov/CompEcon -- this has some nice python notebooks that I could try and replicate.
                        I do not know how to start this though? Should I just start trying to bang away at it?
                        Do I even understand how the method works? Do I solve for optimal policy functions for a given
                  \item Good textbook treatments of SMM are found in Adda and Cooper (2003, pp. 87-100) and Davidson and MacKinnon (2004, pp. 383-394).
                \end{enumerate}
          \item
          \item Medical expense data?
          \item Read \cite{banks_et_al_aej_2019} -- this looks at differences in consumption across retirement in the US and the UK.
          \item Read \cite{gourinchas_parker_ec_2003} -- big contribution to the field it seems.
          \item \textbf{Scrap this idea. Eric advised against it.}
        \end{enumerate}

  \item I am back to the annuity idea. I could document changes in annuities in ELSA and FES and also
        see if consumption changed because of this. I like the idea of fitting a bequests model to the change in annuitisation
        rates like Lee Lockwood did but I do not know if that is possible. What moments would I match it to?

  \item I have just broken my version of R annoyingly.
        I want to have a separate set of packages downloaded for my thesis.
        Done and it is working again nicely now.

\end{itemize}


\bibliographystyle{chicago}
\bibliography{references}
\end{document}