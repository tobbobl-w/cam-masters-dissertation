\documentclass[12pt]{article}
\usepackage{graphicx}
\usepackage{amsmath}
\usepackage{amssymb}
\usepackage[authoryear]{natbib}
\usepackage{enumerate}
\setlength{\parindent}{0pt}
\usepackage[parfill]{parskip}
\date{Lent 2023}
\title{Dissertation notes}
\author{Tobias Leigh-Wood}
\usepackage[letterpaper, margin=1.1in]{geometry}
\linespread{1.1}

\begin{document}
\maketitle

\section*{To Do List}
\begin{itemize}
    \item What actually happened?
          Annuity provision peaked in 2012 (\cite{cannon_effect_2016}) Prior to the March 2011 budget individuals had to annuitise their pension pot.
          After 2011 they only had to annuitise up to a minimum income requirement (MIR) of £20,000. This minimum income kept most people using annuities.

          In March 2014 there was a surprise announcement that the MIR would be scrapped and individuals would be able to 'draw down' their pension pot instead.
          This meant the compulsory market for pension annuities basically stopped. Moreover, the amount needed to be classed as a 'small pot' increased meaning there were more pots that could be taken out all at once.
          In particular this reform meant the number of small annuities in the compulsory market was drastically reduced.


    \item \textbf{When did the changes take place?}

          First round in 2012 and a second round of changes in March 2014. Which age groups were particularly impacted?

    \item \textbf{When were the changes announced?}

          There were expected changes announced in March 2011. Unexpected changes made in March 2014.
          The changes made in 2011 would still have been quite unexpected so we could probably still expect changes in early retirement consumption.

          So perhaps we could look at people just above and just below the cut offs.
    \item \textbf{What were the pension rules before?}
          Were there cutoffs that changed peoples behaviour?

\end{itemize}

\begin{itemize}
    \item I want some graphs with the elsa data showing the number of people with annuities by cohort.
    \item This should come from the pension data i think rather than the financial data.
\end{itemize}

\bibliographystyle{chicago}
\bibliography{references}
\end{document}